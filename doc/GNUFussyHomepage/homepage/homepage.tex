% ******************************************************************
%  * Copyright (c) 2000-2018, 2019 S.Bhatnagar
%  *
% #   This file is part of fussy.
% #
% #   fussy is a free software: you can redistribute it and/or modify
% #   it under the terms of the GNU General Public License as published by
% #   the Free Software Foundation, either version 3 of the License, or
% #   (at your option) any later version.
% #
% #   fussy is distributed in the hope that it will be useful,
% #   but WITHOUT ANY WARRANTY; without even the implied warranty of
% #   MERCHANTABILITY or FITNESS FOR A PARTICULAR PURPOSE.  See the
% #   GNU General Public License for more details.
% #
% #   You should have received a copy of the GNU General Public License
% #   along with fussy.  If not, see <https://www.gnu.org/licenses/>.
%  *
% ******************************************************************
%\documentclass[]{acmtrans2m}
%\documentclass[acmtoms,acmnow]{acmtrans2m}
\documentclass[11pt]{article}
\renewcommand{\familydefault}{\sfdefault}
\usepackage{fancyhdr}     % Use Utopia font - used for printing the thesis
%\usepackage{utopia}     % Use Utopia font
\usepackage{ifpdf}
\usepackage{amsmath}
\usepackage[dvips]{graphicx, color}  % The figure package
%\usepackage{palatino}
%\usepackage{txfonts}     % Use Utopia font
\usepackage{natbib}
\usepackage{marvosym}
\usepackage{xcolor}
\usepackage{txfonts}
\usepackage{enumitem}
\usepackage[breaklinks,colorlinks]{hyperref}

\usepackage{hyperref}
\hypersetup{
    colorlinks,
    citecolor=gray,
    filecolor=black,
    linkcolor=blue,
    urlcolor=red
}

\raggedbottom

\setlength{\textheight}{22.0cm}
\setlength{\textwidth}{16.00cm}
\setlength{\topmargin}{-0.5cm}
\setlength{\oddsidemargin}{0.0cm}
\setlength{\evensidemargin}{0.5cm}
\setlength{\parskip}{5pt}
\setlength{\parindent}{20pt}

\newcommand{\Fussy}{GNU {\tt fussy}}
\newcommand{\DS}{{\tt DS}}
\newcommand{\VMS}{{\tt VMS}}
\newcommand{\PDFVersion}  {\htmladdnormallinkfoot{PDF version}{http://www.aoc.nrao.edu/~sbhatnag/Softwares/fussy/fussy.pdf}}
           
\pagestyle{fancyplain}
\rhead[]{Language with automatic error propagation}      % Note the different brackets!
\lhead[]{{\tt fussy}}
% \pagestyle{myheadings}
% %\markright{S. Bhatnagar}
% %\pagestyle{headings}
% \markboth{S. Bhatnagar}{Automatic random error propagation: S. Bhatnagar}


\begin{document}
\title{{\Fussy}: A language with automatic error propagation}
\author{S. Bhatnagar}
%\date{Nov. 2003}
\date{}
%\maketitle
\normalsize
%% \ifpdf
%% \else
%% \begin{center}
%%   \PDFVersion
%% \end{center}
%% \fi


\section*{fussy}
The \Fussy\ is a scripting language with automatic propagation of random
measurement errors in arbitrary mathematical expressions.  It is internally implemented
as a virtual machine for efficient runtime performance.  Mathematical
expressions can be implemented as a collection of sub-expressions, as sub-program units
(functions or procedures) or as single atomic expressions.  Errors are correctly
propagated when a complex expression is broken up into smaller sub-expressions.
Sub-expressions are assigned to temporary variables which can then be used to write the
final expression.  These temporary variables are not independent variables and the
information about their dependence on other constituent independent variables is
preserved for error propagation.

The scripting syntax of \Fussy\ is similar to that of {\tt C}.  It is therefore easy to
use with minimal learning and can be used in every day scientific work.  Most other
related work found in the literature is in the form of libraries for automatic
differentiation.  Only two tools appear to have used it for automatic error propagation.
Use of these libraries and tools require sophisticated programming and are targeted more
for programmers than for regular every day scientific use.  Also, such libraries and
tools are difficult to use for correct error propagation in expressions composed of
sub-expressions.

\section*{Downloading}
Currently the best way to download \Fussy\ is to clone it's git repository using the
following command:
\begin{verbatim}
     git clone https://git.savannah.gnu.org/git/fussy.git
\end{verbatim}
This will create a directory named ``fussy''.  The {\tt fussy} can be compiled on most
GNU Linux, Linux-like OSes by running the {\tt build} script in this directory.


\section*{Documentation}
Online manual can be found at https://www.gnu.org/software/fussy/manual.  The manual as
man-pages, {\it info} pages are being written.

After starting {\tt fussy}, the {\tt help} command also provides brief help about
available features and {\tt fussy} language syntax.


\section*{Maintainer}
\Fussy\ is currently maintained by Sanjay Bhatnagar and Daniel K. Lyons.


\end{document}
